\chapter{Introduction to php}

PHP is a server-side scripting language designed for web development but also used as a general-purpose programming language. As of January 2013, PHP was installed on more than 240 million websites (39 percent of those sampled) and 2.1 million web servers. Originally created by Rasmus Lerdorf in 1994, the reference implementation of PHP (powered by the Zend Engine) is now produced by The PHP Group. While PHP originally stood for Personal Home Page, it now stands for PHP: Hypertext Preprocessor, which is a recursive backronym.

PHP code can be simply mixed with HTML code, or it can be used in combination with various templating engines and web frameworks. PHP code is usually processed by a PHP interpreter, which is usually implemented as a web server's native module or a Common Gateway Interface (CGI) executable. After the PHP code is interpreted and executed, the web server sends the resulting output to its client, usually in the form of a part of the generated web page; for example, PHP code can generate a web page's HTML code, an image, or some other data. PHP has also evolved to include a command-line interface (CLI) capability and can be used in standalone graphical applications.

The standard PHP interpreter, powered by the Zend Engine, is free software released under the PHP License. PHP has been widely ported and can be deployed on most web servers on almost every operating system and platform, free of charge.

Despite its popularity, no written specification or standard existed for the PHP language until 2014, leaving the canonical PHP interpreter as a de facto standard. Since 2014, there is ongoing work on creating a formal PHP specification.

\section{PHP introduction}

Since we assume you are familiar with other programming languages, we will only go over the basics of PHP. For this we refer to the powerpoint presentation \url{Intro_php.pptx}.